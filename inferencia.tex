\documentclass[10pt,landscape]{article}
\usepackage{multicol}
\usepackage{calc}
\usepackage{ifthen}
\usepackage[landscape]{geometry}
\usepackage{graphicx}
\usepackage{amsmath, amssymb, amsthm}
\usepackage{latexsym, marvosym}
\usepackage{pifont}
\usepackage{lscape}
\usepackage{array}
\usepackage{booktabs}
\usepackage[bottom]{footmisc}
\usepackage{tikz}
\usetikzlibrary{shapes}
\usepackage{pdfpages}
\usepackage{wrapfig}
\usepackage{enumitem}
\setlist[description]{leftmargin=0pt}
\usepackage{xfrac}
\usepackage[pdftex,
            pdfauthor={Your Name},
            pdftitle={Template Example},
            pdfsubject={Template showcasing cheat sheet patterns},
            pdfkeywords={template} {example} {cheatsheet}
            ]{hyperref}
\usepackage[
            open,
            openlevel=2
            ]{bookmark}
\usepackage{relsize}
\usepackage{rotating}

% Custom commands from original
\newcommand\independent{\protect\mathpalette{\protect\independenT}{\perp}}
\def\independenT#1#2{\mathrel{\setbox0\hbox{$#1#2$}%
\copy0\kern-\wd0\mkern4mu\box0}}

\newcommand{\noin}{\noindent}
\newcommand{\logit}{\textrm{logit}}
\newcommand{\var}{\textrm{Var}}
\newcommand{\cov}{\textrm{Cov}}
\newcommand{\corr}{\textrm{Corr}}
\newcommand{\N}{\mathcal{N}}
\newcommand{\Bern}{\textrm{Bern}}
\newcommand{\Bin}{\textrm{Bin}}
\newcommand{\Beta}{\textrm{Beta}}
\newcommand{\Gam}{\textrm{Gamma}}
\newcommand{\Expo}{\textrm{Expo}}
\newcommand{\Pois}{\textrm{Pois}}
\newcommand{\Geom}{\textrm{Geom}}
\newcommand{\HGeom}{\textrm{HGeom}}
\newcommand{\NBin}{\textrm{NBin}}

% Geometry settings
\geometry{top=1cm,left=1cm,right=1cm,bottom=1cm}
\pagestyle{empty}
\makeatletter
\renewcommand{\section}{\@startsection{section}{1}{0mm}%
                                {-1ex plus -.5ex minus -.2ex}%
                                {0.5ex plus .2ex}%x
                                {\normalfont\large\bfseries}}
\renewcommand{\subsection}{\@startsection{subsection}{2}{0mm}%
                                {-1explus -.5ex minus -.2ex}%
                                {0.5ex plus .2ex}%
                                {\normalfont\normalsize\bfseries}}
\renewcommand{\subsubsection}{\@startsection{subsubsection}{3}{0mm}%
                                {-1ex plus -.5ex minus -.2ex}%
                                {1ex plus .2ex}%
                                {\normalfont\small\bfseries}}
\makeatother
\setcounter{secnumdepth}{0}
\setlength{\parindent}{0pt}
\setlength{\parskip}{0pt plus 0.5ex}

\begin{document}
\raggedright
\footnotesize
\begin{multicols}{3}

\setlength{\premulticols}{1pt}
\setlength{\postmulticols}{1pt}
\setlength{\multicolsep}{1pt}
\setlength{\columnsep}{2pt}


\section{Convergência}\smallskip \hrule height 2pt \smallskip

\subsection{Convergência quase certa}
    \begin{description}
    \end{description}

\subsection{Convergência em probabilidade}
    \begin{description}
    \end{description}

\subsection{Convergência em distribuição}
    \begin{description}
    \end{description}

\section{Estimação por Máxima Verossimilhança}\smallskip \hrule height 2pt \smallskip

\subsection{Propriedades}
    \begin{description}
    \end{description}

\subsection{Cálculo da EMV}
    \begin{description}
    \end{description}

\subsection{Propriedades Assintóticas}
    \begin{description}
    \end{description}

\section{Estimadores de Bayes}\smallskip \hrule height 2pt \smallskip

\subsection{Propriedades}
    \begin{description}
    \end{description}

\subsection{Cálculo do EB}
    \begin{description}
    \end{description}

\subsection{Estimadores sob funções de perda}
    \begin{description}
    \end{description}

\section{Estatísticas Suficientes}\smallskip \hrule height 2pt \smallskip



\end{multicols}
\end{document}
