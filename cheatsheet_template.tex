\documentclass[10pt,landscape]{article}
\usepackage{multicol}
\usepackage{calc}
\usepackage{ifthen}
\usepackage[landscape]{geometry}
\usepackage{graphicx}
\usepackage{amsmath, amssymb, amsthm}
\usepackage{latexsym, marvosym}
\usepackage{pifont}
\usepackage{lscape}
\usepackage{array}
\usepackage{booktabs}
\usepackage[bottom]{footmisc}
\usepackage{tikz}
\usetikzlibrary{shapes}
\usepackage{pdfpages}
\usepackage{wrapfig}
\usepackage{enumitem}
\setlist[description]{leftmargin=0pt}
\usepackage{xfrac}
\usepackage[pdftex,
            pdfauthor={Your Name},
            pdftitle={Cheat Sheet Template},
            pdfsubject={Template showcasing cheat sheet patterns},
            pdfkeywords={template} {cheatsheet} {statistics}
            ]{hyperref}
\usepackage[
            open,
            openlevel=2
            ]{bookmark}
\usepackage{relsize}
\usepackage{rotating}

% Custom commands from original
\newcommand\independent{\protect\mathpalette{\protect\independenT}{\perp}}
\def\independenT#1#2{\mathrel{\setbox0\hbox{$#1#2$}%
\copy0\kern-\wd0\mkern4mu\box0}}

\newcommand{\noin}{\noindent}
\newcommand{\logit}{\textrm{logit}}
\newcommand{\var}{\textrm{Var}}
\newcommand{\cov}{\textrm{Cov}}
\newcommand{\corr}{\textrm{Corr}}
\newcommand{\N}{\mathcal{N}}
\newcommand{\Bern}{\textrm{Bern}}
\newcommand{\Bin}{\textrm{Bin}}
\newcommand{\Beta}{\textrm{Beta}}
\newcommand{\Gam}{\textrm{Gamma}}
\newcommand{\Expo}{\textrm{Expo}}
\newcommand{\Pois}{\textrm{Pois}}
\newcommand{\Geom}{\textrm{Geom}}
\newcommand{\HGeom}{\textrm{HGeom}}
\newcommand{\NBin}{\textrm{NBin}}

% Geometry settings
\geometry{top=1cm,left=1cm,right=1cm,bottom=1cm}
\pagestyle{empty}
\makeatletter
\renewcommand{\section}{\@startsection{section}{1}{0mm}%
                                {-1ex plus -.5ex minus -.2ex}%
                                {0.5ex plus .2ex}%
                                {\normalfont\large\bfseries}}
\renewcommand{\subsection}{\@startsection{subsection}{2}{0mm}%
                                {-1explus -.5ex minus -.2ex}%
                                {0.5ex plus .2ex}%
                                {\normalfont\normalsize\bfseries}}
\renewcommand{\subsubsection}{\@startsection{subsubsection}{3}{0mm}%
                                {-1ex plus -.5ex minus -.2ex}%
                                {1ex plus .2ex}%
                                {\normalfont\small\bfseries}}
\makeatother
\setcounter{secnumdepth}{0}
\setlength{\parindent}{0pt}
\setlength{\parskip}{0pt plus 0.5ex}

\begin{document}
\raggedright
\footnotesize
\begin{multicols}{3}

\setlength{\premulticols}{1pt}
\setlength{\postmulticols}{1pt}
\setlength{\multicolsep}{1pt}
\setlength{\columnsep}{2pt}

\section{Basic Concepts}\smallskip \hrule height 2pt \smallskip

\subsection{Fundamental Definitions}
\begin{description}
    \item[Random Variable] A function $X: \Omega \to \mathbb{R}$ that assigns a real number to each outcome in the sample space $\Omega$.
    \item[Distribution] The collection of all possible values and their probabilities. We write $X \sim F$ for "$X$ follows distribution $F$."
    \item[Independence] Events $A$ and $B$ are independent if:
    \begin{align*}
        P(A \cap B) &= P(A)P(B) \\
        P(A|B) &= P(A) \\
        A \independent B
    \end{align*}
\end{description}

\subsection{Key Probability Rules}
\begin{description}
    \item[Bayes' Theorem]
    \[P(A|B) = \frac{P(B|A)P(A)}{P(B)}\]

    \item[Law of Total Probability] For partition $\{B_i\}$:
    \[P(A) = \sum_{i} P(A|B_i)P(B_i)\]

    \item[Inclusion-Exclusion]
    \[P(A \cup B) = P(A) + P(B) - P(A \cap B)\]
\end{description}

\section{Expected Value and Variance}\smallskip \hrule height 2pt \smallskip

\subsection{Expected Value}
\begin{minipage}{\linewidth}
    \centering
    % Replace with actual figure path when available
    % \includegraphics[width=1.5in]{figures/expectation.pdf}
    \textbf{[Expected Value Visualization]}
\end{minipage}

\begin{description}
    \item[Definition] The average value of a random variable:
    \[E(X) = \sum_{x} x \cdot P(X = x) \text{ (discrete)}\]
    \[E(X) = \int_{-\infty}^{\infty} x f(x) dx \text{ (continuous)}\]

    \item[Linearity] For random variables $X, Y$ and constants $a, b, c$:
    \[E(aX + bY + c) = aE(X) + bE(Y) + c\]

    \item[Indicator Variables] For event $A$:
    \[E(I_A) = P(A)\]
\end{description}

\subsection{Variance and Standard Deviation}
\begin{description}
    \item[Variance] Measures spread around the mean:
    \[\var(X) = E(X^2) - [E(X)]^2\]

    \item[Standard Deviation]
    \[\text{SD}(X) = \sqrt{\var(X)}\]

    \item[Properties]
    \begin{itemize}
        \item $\var(aX + b) = a^2\var(X)$
        \item If $X \independent Y$: $\var(X + Y) = \var(X) + \var(Y)$
    \end{itemize}
\end{description}

\section{Common Distributions}\smallskip \hrule height 2pt \smallskip

\subsection{Binomial Distribution}
\begin{minipage}{\linewidth}
    \centering
    % Replace with actual figure path when available
    % \includegraphics[width=1.2in]{figures/binomial.pdf}
    \textbf{[Binomial PMF Plot]}
\end{minipage}

Let $X \sim \Bin(n, p)$. We know:

\begin{description}
    \item[Story] $X$ counts successes in $n$ independent trials, each with success probability $p$.

    \item[Example] Coin flips: 10 flips of fair coin gives $\Bin(10, 0.5)$. Basketball: 15 free throws with 80\% success rate gives $\Bin(15, 0.8)$.

    \item[PMF]
    \[P(X = k) = \binom{n}{k} p^k (1-p)^{n-k}\]

    \item[Properties]
    \begin{itemize}
        \item $E(X) = np$
        \item $\var(X) = np(1-p)$
        \item Sum property: $\Bin(n,p) + \Bin(m,p) = \Bin(n+m,p)$ (if independent)
        \item Normal approximation: $\Bin(n,p) \approx \N(np, np(1-p))$ for large $n$
    \end{itemize}
\end{description}

\subsection{Poisson Distribution}
Let $X \sim \Pois(\lambda)$. We know:

\begin{description}
    \item[Story] $X$ counts rare events in a fixed interval, occurring at average rate $\lambda$.

    \item[Example] Emails per hour, accidents per day, typos per page. Server crashes averaging 3 per month: $\Pois(3)$.

    \item[PMF]
    \[P(X = k) = \frac{\lambda^k e^{-\lambda}}{k!}\]

    \item[Properties]
    \begin{itemize}
        \item $E(X) = \lambda$
        \item $\var(X) = \lambda$
        \item Sum property: $\Pois(\lambda) + \Pois(\mu) = \Pois(\lambda + \mu)$ (if independent)
        \item Poisson approximation: $\Bin(n,p) \approx \Pois(np)$ when $n$ large, $p$ small
    \end{itemize}
\end{description}

\subsection{Normal Distribution}
Let $X \sim \N(\mu, \sigma^2)$. We know:

\begin{description}
    \item[Story] $X$ is a continuous random variable with bell-shaped distribution, symmetric around $\mu$.

    \item[PDF]
    \[f(x) = \frac{1}{\sqrt{2\pi\sigma^2}} e^{-\frac{(x-\mu)^2}{2\sigma^2}}\]

    \item[Properties]
    \begin{itemize}
        \item $E(X) = \mu$
        \item $\var(X) = \sigma^2$
        \item Linear transformation: $aX + b \sim \N(a\mu + b, a^2\sigma^2)$
        \item Sum property: $\N(\mu_1, \sigma_1^2) + \N(\mu_2, \sigma_2^2) = \N(\mu_1 + \mu_2, \sigma_1^2 + \sigma_2^2)$ (if independent)
    \end{itemize}
\end{description}

\section{Problem-Solving Strategies}\smallskip \hrule height 2pt \smallskip

\subsection{Common Techniques}
\begin{description}
    \item[Linearity of Expectation] Break complex $X$ into sum: $X = X_1 + X_2 + \cdots + X_n$ where $X_i$ are simpler.

    \item[Indicator Variables] For counting problems, use $I_A = 1$ if event $A$ occurs, 0 otherwise. Then $E(I_A) = P(A)$.

    \item[Conditioning] When stuck, condition on useful event $B$:
    \[P(A) = P(A|B)P(B) + P(A|B^c)P(B^c)\]
    \[E(X) = E(X|B)P(B) + E(X|B^c)P(B^c)\]

    \item[Symmetry] Look for symmetric situations to simplify calculations.

    \item[LOTUS] Law of the Unconscious Statistician:
    \[E(g(X)) = \sum_x g(x)P(X = x)\]
\end{description}

\subsection{Distribution Recognition}
\begin{description}
    \item[Binomial] Fixed number of independent trials, each success/failure
    \item[Geometric] Number of failures before first success
    \item[Poisson] Rare events in fixed interval
    \item[Hypergeometric] Sampling without replacement
    \item[Normal] Continuous, bell-shaped, often from CLT
\end{description}

\end{multicols}
\end{document}
